\documentclass[12pt]{article}
\begin{document}
\title{A Latex Report of Evolution of Modern Health Care System}
\maketitle

\subsection{Introduction}
As hospitals, physician groups and health systems continue to consolidate, navigating a sustainable route to clinical and operational alignment is more complex. While systems search for ways to scale up, many organizations are finding that partnerships are an easier choice. Providers increasingly look to team up with other healthcare providers to streamline operations and minimize risk rather than trying to control each lever of the care continuum.

The skills needed to lead a health system through cultural and operational change are evolving along with healthcare business model. As systems acquire more types of service lines and turn to , the hospital is becoming less of a focal point.

Healthcare's push toward greater coordination and integration has required a change in perspective and leadership strategy. Executives increasingly view their organizations as a clinic that operates hospitals, rather than a hospital-centered system.

\subsection{Improvement in Healthcare}
Since 2000, more and more initiatives have been taken at the international and national levels in order to strengthen national health systems as the core components of the global health system. Having this scope in mind, it is essential to have a clear, and unrestricted, vision of national health systems that might generate further progress in global health. The elaboration and the selection of performance indicators are indeed both highly dependent on the conceptual framework adopted for the evaluation of the health systems performance.Like most social systems, health systems are complex adaptive systems where change does not necessarily follow rigid management models. In complex systems path dependency, emergent properties and other non-linear patterns are seen, which can lead to the development of inappropriate guidelines for developing responsive health systems.


Health Policy and Systems Research (HPSR) is an emerging multidisciplinary field that challenges disciplinary capture by dominant health research traditions, arguing that these traditions generate premature and inappropriately narrow definitions that impede rather than enhance health systems strengthening.HPSR focuses on low- and middle-income countries and draws on the relativist social science paradigm which recognises that all phenomena are constructed through human behaviour and interpretation. In using this approach, HPSR offers insight into health systems by generating a complex understanding of context in order to enhance health policy learning.HPSR calls for greater involvement of local actors, including policy makers, civil society and researchers, in decisions that are made around funding health policy research and health systems strengthening.

\subsection{Conclusion}
Societal changes and scientific advances throughout history have brought about enormous improvements in the achievement of health. Today, an optimized level of “health,” whatever the definition might be, is fathomable and achievable if given unlimited resources. The problem lies in that resources are not unlimited, and are in fact disproportionately allocated between demographic groups. A value-based system, designed to provide a high quality of healthcare for the lowest cost, is a solution to the growing crisis of healthcare systems.A major problem with value-based care, however, is that these health outcomes are subjective and determined by individual patient needs and values. A new definition of health, which  incorporates a description of well-being, specific patient needs, and the organizational, value-based system required to satisfy those needs, is now necessary.

Twenty-first-century technology promises to continue changing the nature, complexity, and costs of healthcare. As knowledge increases about the genetic bases of disease, the healthcare system will make greater use of gene therapies, developing ways to prevent genetically caused diseases. Just as the impact of new technologies, such as x rays, antibiotics, vaccines, and surgical advances changed early and mid-twentieth-century medicine socially and scientifically, scientific and medical innovations, as well as social movements and economic realities, will continue to shape twenty-first-century medicine and health care.    
 
\end{document}
